\subsubsection{The singular form $\tilde{\psi}_1$}


We define the singular form $\tilde{\psi}_1$ by
\begin{align*}\label{GreeneqV}
\tilde{\psi}_1(x)  &= - \left( \int_1^{\infty} \psi_1^0(\sqrt{r}x)  \frac{dr}{r} \right)e^{-\pi(x,x)} = - \frac1{2\pi(x_3^2+x_4^2)} \psi_1(x). 
\end{align*}
for $x\ne 0$, and as before $\tilde{\psi}^0_{2,0}(x) = \tilde{\psi}_1(x) e^{\pi (x,x)}$ and 
$\tilde{\psi}_1(x,z)$. We see that $\tilde{\psi}_1$ is defined for
$x \notin \Span[e_3,e_4]^{\perp}$. Formulated differently,
$\tilde{\psi}_1(x,z)$ for fixed $x$ is defined for $z \notin D_x$.
Furthermore, as if $\tilde{\psi}_1$ was a Schwartz function of
weight $2$, we define
\begin{align*}\label{xiZV}
\tilde{\psi}_1(x,\tau,z) &=  \tilde{\psi}_1^0(\sqrt{v}x,z) e^{\pi i (x,x)\tau}  = - \left( \int_v^{\infty} \psi_1^0(\sqrt{r}x,z) \frac{dr}{r} \right) e^{\pi i (x,x)\tau}.
\end{align*}

\begin{proposition}\label{schluesselV}

$\tilde{\psi}_1(x,z)$ is a \bf{differential} $1$-form with singularities
along $D_x$. Outside $D_{x}$, we have
\[ d\tilde{\psi}_1(x,z) = \varphi_2(x,z).  \]
Here $d$ denotes the exterior differentiation on $D$. In particular,
for $(x,x)\leq 0$, we see that $\varphi_2(x)$ is exact. 
Furthermore,
\[ L\tilde{\psi}_1(x,\tau) = \psi_1(x,\tau).  \]
\end{proposition}

\begin{proof}

Using \eqref{GreeneqV} and \eqref{partial-d}, we see
\begin{align*}
d \tilde{\psi}_1^0(x,z) &= - \int_1^{\infty}d \left(\psi_1^0(\sqrt{r}x,z)\right)\frac{dr}{r}   =-\int_1^{\infty} \frac{\partial}{\partial r } \left(
\varphi_2^0(\sqrt{r}x,z)\right) \frac{dr}{r}  = \varphi_2^0(x,z),
 \end{align*}
as claimed. The formula $L\tilde{\psi}_1(x,\tau) = \psi_1(x,\tau)$
follows easily from \eqref{xiZV}.
\end{proof}




\begin{remark}
The construction of the singular form $\tilde{\psi}$ works in much greater generality for $\Orth(p,q)$ whenever we have two Schwartz forms $\psi$ and $\varphi$ (of weight $r-2$ and $r$ resp.) such that 
\[
d \psi = L \varphi.
\]
Then the analogous construction of $\tilde{\psi}$ then immediately yields $d \tilde{\psi} = \varphi$ outside a singular set. The main example for this are the general Kudla-Millson forms $\varphi_{q}$ and $\psi_{q-1}$, see \cite{KM90}. For these forms, this construction is already implicit in \cite{BFDuke}. In particular, the proof of Theorem~7.2 in \cite{BFDuke} shows that $\tilde{\psi}$ gives rise to a differential character for the analogous cycle $C_x$, see also Section~\ref{currents} of this paper. The unitary case will be considered in \cite{F-unitary}. 
\end{remark}


